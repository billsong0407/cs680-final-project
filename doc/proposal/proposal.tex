\documentclass[12pt]{article}
\usepackage[utf8]{inputenc}
\usepackage{fullpage}
\setlength\parindent{0pt}
\usepackage{amsmath,amsthm,amssymb, cancel, hyperref}


\title{%
  Detecting Plant Diseases: A Comprehensive Approach to Houseplant Health }

\author{Bill Song - b26song@uwaterloo.ca\\Raymond Liu - r354liu@uwaterloo.ca}

\date{October 8, 2024}

\begin{document}

\maketitle



\section{Overview}

House plants have become an important part of many households, providing many benefits such as enhancing air quality and lowering stress. However, house plants are vulnerable to many diseases, and many people lack the knowledge to identify the health issues of their plants. This project addresses this problem by designing a machine learning model that analyzes photos of house plants and identifies the specific type of disease affecting this house plant (or that it is healthy). While deep learning has been widely used for plant disease detection, most existing efforts focus on agricultural crops. This project, however, is specifically tailored to house plants, addressing a gap in current research. The input images are allowed to be any type of house plant, as we will try to include as many house plant species in our dataset as possible. The classification, however, will be focused on three common house plant diseases -- anthracnose, powdery mildew, sooty mold, (and healthy) -- for now, due to the large number of plant diseases. It is possible that we extend our disease classes as the project progresses. 




\section{Related Works}
\begin{itemize}
    \item Performance of deep learning vs machine learning in plant leaf disease detection \cite{Sujatha2020}: This paper compares different types of machine learning and deep learning techniques for classifying citrus diseases. It examines four types of diseases for citrus: black spot, melanose,  canker, and greening. The ML and DL methods being applied are SGD, RF, SVM, Inception-v3, VGG-16, and VGG-19. The paper uses a confusion matrix as its metric and evaluates each method based on accuracy, recall, precision, and F1 score. They achieved accuracy scores ranging from 76\% to close to 90\%, with VGG-19 having the best performance. 

    \item Houseplant leaf classification system based on deep learning algorithms \cite{Hama2024}:
    This paper focuses on house plant classification. It uses a dataset of 2500 images, where 250 images for each class. The images were pre-processed into 224x224x3 and the dataset was divided into 8:1:1 for training, validating, and testing. They also augmented the dataset by rotating, flipping, and shifting. Their choice of model to fine-tune is ResNet-50. The authors trained their dataset on ResNet in different stages, where at stage i, they freeze the i-th block of the model and train the remaining part. Their metrics were based on the confusion matrix and included accuracy, recall, precision, and F1-score. They were able to achieve close to perfect classification results (98.6\%, 99\%, etc).

    \item Disease detection on the leaves of the tomato plants by using deep learning \cite{Durmus2017}: This paper detects tomato plant disease in real-time using deep learning. They extracted photos of tomato plant leaves from PlantVillage, a dataset that contains 54.309 labeled images for 14 different crops, and applied AlexNet and SqueezeNet to their extracted dataset. They were able to obtain an accuracy score of 95\% on AlexNet and 94\% on SqueezeNet on the test dataset. No confusion matrix or other evaluation metric was used.

    
\end{itemize}

\textbf{Summary}: None of the related works directly address the task of classifying house plant diseases. The first paper surveys various machine learning and deep learning techniques for citrus disease classification, and demonstrated that deep learning techniques -- VGG-19 -- achieved the best results. The similarity between house plant disease and citrus disease, suggests that using deep learning techniques for our project will indeed give better results than other techniques. The second paper focuses on classifying house plant types and achieves near-perfect results using ResNet-50 with advanced training techniques. Inspired by their results, we also choose ResNet-50 as our model choice and plan to evaluate our results based on the confusion matrix. The final paper examines the accuracy of deep learning models in detecting tomato plant diseases, using the relatively old AlexNet model. No specific training techniques were mentioned in their paper, but it provides a good example on real-time plant disease classification using deep learning, which we might reference if we also decided to implement real-time classification.


\section{Execution Plan}
\subsection{Data Collection}
We plan to create a dataset of 2000 images, with 500 images for each category (anthracnose, powerdry mildew, sooty mold, healthy). The images will mostly be obtained through web scraping on popular search engines such as Bing and Google, with keywords in the form of "plant\_name + disease". We will include a wide variety of house plant species for each category in our dataset. Besides web scraping, we also plan to extract house plant images from Kaggle such as  \cite{HPSD} and \cite{HWHP} to form a dataset for the healthy category.




\subsection{Data Preprocessing}
The collected images will then be preprocessed to maintain consistency across the dataset. Steps will include resizing the images to a standard size suitable for the model, as well as labeling the collected datasets into their respective categories (diseased vs. healthy). Moreover, image transformations (e.g., rotations, flips, and brightness adjustments) will be employed to augment the dataset and improve model accuracy and robustness during training.

\subsection{Model Training}
The dataset will be split into training, validation, and testing datasets following an 8:1:1 ratio. We plan to fine-tune ResNet50 to detect house plant diseases. ResNet50 can handle complex image classification tasks while maintaining high accuracy through deep feature extraction, and it does not need bounding boxes on the dataset. The model will be fine-tuned using transfer learning and adjusting pre-trained weights from ImageNet to detect specific plant diseases. The model’s hyperparameters, such as learning rate and batch size, will be carefully optimized during the training process. The network architecture will retain ResNet50's core layers while adjusting the output layer to match the four classification categories (three diseases + healthy).

\subsection{Model Evaluation}
The model’s performance will be evaluated using the validation set during training and further evaluated on the testing set after all training iterations are completed. We will adopt classification evaluation metrics such as confusion matrix, accuracy, precision, recall, and F1 score. These evaluation metrics are good indicators to tell whether the model effectively distinguishes between each diseases and healthy plants. 


\section{Evaluation Plan}
\subsection{Confusion Matrix}
A confusion matrix will be constructed to analyze the model's performance across all four classes (three diseases + healthy). This will allow us to visualize the number of true positives, false positives, true negatives, and false negatives for each category.

\subsection{Test Dataset Accuracy}
The model's accuracy such as precision, recall, and F1 score will be calculated on the test dataset to quantify how well it classifies unseen data. In addition, ROC curves will be plotted for each class, with the area under the curve (AUC) used to assess the ability of the model to distinguish between classes.

\subsection{Manual Testing with Live Photos}
Finally, the model will undergo manual testing using live plant photos to verify its real-world performance, ensuring that it generalizes well outside the dataset.



\section{Expected Outcomes / Contribution Objectives}
\subsection{Accuracy}
The resulting confusion matrices, ROC (Receiver Operating Characteristic) curves and the corresponding AUC (Area Under the ROC Curve) scores should provide us with detailed quantitative information to verify whether the model can reliably distinguish between different plant diseases so that the false positive and/or false negative rates are minimized. Based on these metrics, the deep learning model will allow us to group plant images into four choices – anthracnose, powdery mildew, sooty mold, and a healthy plant. If the fine-tuned ResNet50 model is well-trained and the dataset is well-designed and prepared, the classification accuracy on the test dataset should be above 90\%. The accuracy, precision, recall, and F1-score based on the confusion matrix are also expected to be above 90\%.





\begin{thebibliography}{9}

\bibitem{Sujatha2020}
R. Sujatha, J. Chatterjee, NZ Jhanjhi, S. Brohi 
\textit{Performance of deep learning vs machine learning in plant leaf
disease detection}. 
Microprocessors and Microsystems, 2020. https://www.sciencedirect.com/science/article/pii/S0141933120307626

\bibitem{Hama2024}
H. Hama, T. Absulsamad, S. Omer
\textit{Houseplant leaf classification system based on deep learning algorithms}. 
Journal of Electrical System and Information Technology, vol. 11, 2024. https://link.springer.com/article/10.1186/s43067-024-00141-5

\bibitem{Durmus2017}
H. Durmus, E. Gunes, M. Kirci
\textit{Disease detection on the leaves of the tomato plants by using deep learning}. 
IEEE, 2017. https://ieeexplore.ieee.org/abstract/document/8047016

\bibitem{KaggleCode}
Evaluation metrics for multi-class classification: 

https://www.kaggle.com/code/nkitgupta/evaluation-metrics-for-multi-class-classification


\bibitem{HPSD}
House Plant Species Dataset:
https://www.kaggle.com/datasets/kacpergregorowicz/house-plant-species/data

\bibitem{HWHP}
Healthy and Wilted Houseplant Images: https://www.kaggle.com/datasets/russellchan/healthy-and-wilted-houseplant-images

\end{thebibliography}




\end{document}